%
% fig-wurzel.tex
%
% (c) 2026 Prof Dr Andreas Müller
%
\begin{figure}
\centering
\includegraphics[width=\textwidth]{chapters/040-analytisch/images/wurzel.pdf}
\caption{Die analytische Fortsetzung der Wurzelfunktion entlang eines
Pfades, der den Nullpunkt umschliesst, führt auf eine widersprüchliche
Definition.
Damit eine Wurzelfunktion als holomorphe Funktion definiert werden kann,
dann muss der Definitionsbereich zur Riemann-Fläche $M$ ``aufgedoppelt''
werden.
Die beiden Blätter der Riemann-Fläche schaffen Raum für die beiden
möglichen Werte, die $\!\sqrt{z\mathstrut}$ annehmen kann.
\label{buch:analytisch:fortsetzung:fig:wurzel}}
\end{figure}
