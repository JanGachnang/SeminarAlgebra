%
% 3-fortsetzung.tex -- Analytische Fortsetzung
%
% (c) 2025 Prof Dr Andreas Müller
%
\section{Analytische Fortsetzung
\label{buch:analytisch:section:fortsetzung}}
\kopfrechts{Analytische Fortsetzung}
Auf den ersten Blick scheint die Beschreibung einer holomorphen
Funktion durch eine Potenzreihe einen Verlust zu beinhalten.
Eine Potenzreihe ist ja nur innerhalb des Konvergenzkreises
konvergent, während eine holomorphe Funktion einen viel grösseren
Definitionsbereich haben kann.
Das Definitionsgebiet lässt sich aber immer mit Kreisen überdecken,
innerhalb derer sich die Funktion durch eine Potenzreihe schreiben
lässt.
In den Überschneidungsgebieten dieser Kreise stimmen die Funktionen
überein.
Der in diesem Abschnitt dargestellte Prozess der analytischen
Fortsetzung ermöglicht, eine holomorphe Funktion entlang einer
Kette von Kreisgebieten zu erweitern.
So lässt sich das grösstmögliche Definitionsgebiet
für eine holomorphe Funktion finden.
Es zeigt aber zum Beispiel bei der Logarithmus-Funktion auch, dass
die Menge der komplexen Zahlen zu klein sein kann und durch eine
Überlagerung ersetzt werden muss, um eine wohldefinierte komplex
differenzierbare Funktion zu konstruieren.

%
% Verschiedene Definitionsgebiete
%
\subsection{Verschiedene Definitionsgebiete}
Wir betrachten zwei Potenzreihen
\[
f_0(z)
=
\sum_{k=0}^\infty a^0_k(z-z_0)^k
\qquad\text{und}\qquad
f_1(z)
=
\sum_{k=0}^\infty a^1_k(z-z_1)^k
\]
mit Konvergenzradien $\varrho_0$ und $\varrho_1$.
Wir nehmen an, dass
\[
|z_1-z_0| < \varrho_0
\]
so dass der Punkt $z_1$ im Konvergenzkreis der Potenzreihe
$f_0(z)$ liegt.
Die Koeffizienten $a^1_k$ sind vollständig durch die Werte in
einer Umgebung von $z_1$ gegeben.
Um dies einzusehen, schreiben wir $z=(z-z_1) + (z_1-z_0)$ und drücken
\begin{align}
f_0(z)
&=
\sum_{n=0}^\infty a^0_n(z-z_0)^n
\notag
\intertext{durch $z-z_1$ aus und erhalten}
&=
\sum_{n=0}^\infty a^0_n((z-z_1)+(z_1-z_0))^n.
\notag
\intertext{Mit dem Binomialsatz kann dies in}
&=
\sum_{n=0}^\infty \sum_{k+l=n} a^0_n\binom{n}{k}(z_1-z_0)^{l} (z-z_1)^k 
\notag
\intertext{expandiert werden.
Da die innere Summe endlich ist, lässt sich die Summationsreihenfolge
ändern:
}
&=
\sum_{k=0}^\infty
\Biggl(
\sum_{n=k}^\infty
a^0_n\binom{n}{n-k}
(z_1-z_0)^{n-k}
\Biggr)
(z-z_1)^k
\notag
\intertext{Die Binomialkoeffizienten können als Bruch geschrieben werden,
was}
&=
\sum_{k=0}^\infty
\Biggl(
\sum_{n=k}^\infty
a^0_n
\frac{n\cdot(n-1)\cdot\ldots\cdot(n-k+1)}{k!}
(z_1-z_0)^{n-k}
\Biggr)
(z-z_1)^k
\notag
\intertext{ergibt.
Der Nenner $k!$ in den Termen der inneren Summe hängt nicht vom
Summationsindex ab und kann daher vor die Summe genommen werden:}
&=
\sum_{k=0}^\infty
\frac{1}{k!}
\Biggl(
\sum_{n=k}^\infty
a^0_n
\bigl(
n\cdot(n-1)\cdot\ldots\cdot (n-k+1)
\bigr)
(z_1-z_0)^{n-k}
\Biggr)
(z-z_1)^k.
\notag
\intertext{Das Produkt $n\cdot(n-1)\cdot\ldots\cdot(n-k+1)z^{n-k}$
ist die $k$-te Ableitung von $z^n$.
Da $k$ für alle Terme der inneren Summe gleich ist, ist die
innere Summe die $k$-te Ableitung einer einfacheren Summe,
nämlich}
&=
\sum_{k=0}^\infty
\frac{1}{k!}
\frac{d^k}{dz_1^k}
\Biggl(
\sum_{n=k}^\infty
a^0_n
(z_1-z_0)^n
\Biggr)
(z-z_1)^k.
\notag
\intertext{Die innere Summe ist nichts anderes als die Funktion
$f_0(z_1)$:}
&=
\sum_{k=0}^\infty
\frac{1}{k!}
\frac{d^k}{dz_1^k}
f_0(z_1)
(z-z_1)^k.
\label{buch:analytisch:fortsetzung:eqn:taylor01}
\end{align}
Die letzte Reihe ist die Taylor-Reihe der Funktion $f_0$ an der
Stelle $z_1$.

Wir nehmen jetzt an, dass in einer Umgebung vom Radius
$r_1-|z_0|$ die Funktionen $f_0(z)$ und $f_1(z)$ übereinstimmen.
Wir wissen bereits, dass die Koeffizienten $a^i_k$ durch die
Taylor-Reihe der Funktion $f_i(z)$ in den Punkten $z_i$ gegeben
ist.
Das Resultat
\eqref{buch:analytisch:fortsetzung:eqn:taylor01}
der obigen Rechnung bestätigt, durch einen Koeffizientenvergleich,
dass
\[
a^1_k
=
\frac{1}{k!}
\frac{d^kf_0(z_1)}{dz_1^k}.
\]

%
% Analytische Fortsetzung entlang einer Kurve
%
\subsection{Analytische Fortsetzung entlang einer Kurve}

%
% Überlagerungen der komplexen Ebene
%
\subsection{Überlagerungen der komplexen Ebene}
%
% fig-wurzel2.tex
%
% (c) 2026 Prof Dr Andreas Müller
%
\begin{figure}
\centering
\includegraphics{chapters/040-analytisch/images/wurzel2.pdf}
\caption{Die analytische Fortsetzung der Wurzelfunktion
$f(z)=\!\sqrt{z\mathstrut}$
entlang der gelben Kurve führt im blauen Bereich auf eine
widersprüchliche Definition.
Abbildung~\ref{buch:analytisch:fortsetzung:wurzel} zeigt,
wie durch Konstruktion eines grösseren Definitionsbereichs eine
widerspruchsfreie der Wurzelfunktion gewonnen werden kann.
\label{buch:analytisch:fortsetzung:fig:wurzel2}}
\end{figure}

%
% fig-wurzel.tex
%
% (c) 2026 Prof Dr Andreas Müller
%
\begin{figure}
\centering
\includegraphics[width=\textwidth]{chapters/040-analytisch/images/wurzel.pdf}
\caption{Die analytische Fortsetzung der Wurzelfunktion entlang eines
Pfades, der den Nullpunkt umschliesst, führt auf eine widersprüchliche
Definition.
Damit eine Wurzelfunktion als holomorphe Funktion definiert werden kann,
dann muss der Definitionsbereich zur Riemann-Fläche $M$ ``aufgedoppelt''
werden.
Die beiden Blätter der Riemann-Fläche schaffen Raum für die beiden
möglichen Werte, die $\!\sqrt{z\mathstrut}$ annehmen kann.
\label{buch:analytisch:fortsetzung:fig:wurzel}}
\end{figure}

Die analytische Fortsetzung einer holomorphen Funktion entlang eines
Weges kann auf Situationen führen, in denen keine widerspruchsfreie
Definition der Funktion auf dem ganzen Definitionsbereich möglich ist.
Um trotzdem eine konsistente Definition zu ermöglichen, muss der
Definitionsbereich angepasst werden.
Es ist zwar möglich, den Definitionsbereich künstlich zu verkleinern
und auf einen Teil der Funktion zu verzichten.
Doch viel befriedigender ist eine Situation, in der eine Überlagerung
des Definitionsbereichs konstruiert wird, wie dies im Folgenden
gezeigt werden soll.

%
% Definitionsbereich für die Wurzelfunktionen
%
\subsubsection{Definitionsbereich für die Wurzelfunktionen}
Die reelle Wurzelfunktion $x\mapsto \!\sqrt{x\mathstrut}$ lässt
sich in der Polardarstellung $z=r(\cos\varphi+i\sin\varphi)$ einer
komplexen Zahl durch
\[
\sqrt{z\mathstrut}
=
\sqrt{r\mathstrut}
\biggl(\cos\frac{\varphi}2+i\sin\frac{\varphi}2\biggr)
\]
zu einer Wurzelfunktion für komplexe Zahlen erweitert werden.
Die Polardarstellung einer komplexen Zahl ist jedoch nicht eindeutig.
Das Argument $\varphi$ ist nur bis auf ein Vielfaches von $2\pi$
festgelget.
Falls $r$ und $\varphi$ die komplexe Zahl
$
z
=
r(\cos\varphi+i\sin\varphi)
$ 
darstellen, dann gilt auch
$
z
=
r(\cos(\varphi+2\pi k) + i\sin(\varphi+2\pi k)
$.
Die möglichen Werte der Quadratwurzel sind daher
\begin{align*}
\sqrt{z}
&=
\sqrt{r\mathstrut}\biggl(
\cos\frac{\varphi+2\pi k}{2}
+
i
\sin\frac{\varphi+2\pi k}{2}
\biggr)
\\
&=
\sqrt{r\mathstrut}\biggl(
\cos\biggl(\frac{\varphi}2+\pi k\biggr)
+
i
\sin\biggl(\frac{\varphi}2+\pi k\biggr)
\biggr)
\\
&=
\sqrt{r\mathstrut}
(-1)^k
\biggl(\cos\frac{\varphi}2+i\sin\frac{\varphi}2\biggr).
\end{align*}
Wie erwartet hat jede von $0$ verschiedene komplexe Zahl zwei
verschiedene Quadratwurzeln, die sich um den Faktor $-1$ 
unterscheiden.

Die analytische Fortsetzung entlang einer Kurve in der komplexen
Ebene soll helfen, den Wert der Quadratwurzel so festzulegen,
dass eine holomorphe Funktion entsteht.
An der Stelle $z=0$ ist die Quadratwurzel nicht differenzierbar,
da schon die reelle Wurzelfunktion nicht differenzierbar ist.
Wir müssen daher den Punkt $z=0$ aus dem Definitionsbereich
ausschliessen.

Wir betrachten jetzt einen stetigen Weg $\gamma(t)$ in der Menge
$D=\mathbb{C}\setminus\{0\}$, der auf der positiven reellen
Achse beginnt.
Da der Weg nicht durch den Nullpunkt führt, können wir ihn
auch in der Polardarstellung als
\[
\gamma(t)
=
r(t)
\bigl(
\cos\varphi(t)
+
i\sin\varphi(t)
\bigr)
\]
mit stetigen Funktionen $r(t)$ und $\varphi(t)$ geschrieben
werden.
Da die Kurve auf der positiven reellen Achse beginnt, gilt
$\varphi(0)=0$.
Da die Wurzelfunktion $\!\sqrt{x}$ für positive reelle Zahlen $x$
positiv ist, kann der Wert bei analytischer Fortsetzung entlang
der Kurve $\gamma(t)$ nur
\begin{equation}
\!\sqrt{\gamma(t)}
=
\!\sqrt{r(t)}
\biggl(
\cos\frac{\varphi(t)}2
+
i
\sin\frac{\varphi(t)}2
\biggr)
\label{buch:analytisch:fortsetzung:eqn:wurzelfortsetzung}
\end{equation}
sein.

Abbildung~\ref{buch:analytisch:fortsetzung:fig:wurzel2} zeigt die
Konvergenzkreise der analytischen Fortsetzung entlang eines Kreises.
In jedem Kreisgebiet ist die Wurzelfunktion durch 
\eqref{buch:analytisch:fortsetzung:eqn:wurzelfortsetzung}
gegeben.
Entlang der Kurve nimmt das Argument stetig zu, bis sich im
blauen Schnittgebiet in der
Abbildung~\ref{buch:analytisch:fortsetzung:fig:wurzel2}
der Wert des Arguments um $2\pi$ vergrössert hat.
Der Wert der Wurzelfunktion wurde aber bei dieser Argumentänderung
mit $-1$ multipliziert.
Die analytische Fortsetzung entlang der gelben Kurve führt also
zu entgegengesetzen und damit widersprüchlichen Werten, je nachdem wie
oft der Nullpunkt der komplexen Ebene umrundet wurde.



%
% Definitionsbereich für die Logarithmusfunktion
%
\subsubsection{Definitionsbereich für die Logarithmusfunktion}
Wir untersuchen das analoge Problem für die Logarithmusfunktion.
Als Umkehrfunktion der Exponentialfunktion kann sie leicht aus
der Formel
\[
e^z
=
e^x(\cos y + i \sin y)
\]
abgeleitet werden.
Ist $e^z=r(\cos\varphi+i\sin\varphi)$ die Polardarstellung einer
komplexen Zahl, dann muss der Logarithmus $z=x+iy$ die Gleichungen
\begin{align*}
r&=e^x,
&
\cos y &= \cos\varphi
&&\text{und}
\sin y &= \sin\varphi
\end{align*}
erfüllen.
Die erste Gleichung besagt, dass $x=\log r$ die bekannte
reelle Logarithmusfunktion sein muss.
Die letzten zwei Gleichungen besagen, dass sich $\varphi$ und $y$
um ein Vielfaches von $2\pi$ unterscheiden.
Die Menge der möglichen Logarithmuswerte ist daher
\begin{equation}
\{
\log r
+
(\varphi
+
2\pi k)i
\mid
k\in\mathbb{Z}
\}.
\label{buch:analytisch:fortsetzung:eqn:logfort}
\end{equation}

Auch im Fall der Logartihmusfunktion muss daher die analytische
Fortsetzung dazu herangezogen werden, den Wert zu definieren.
Da der Logarithmus im Nullpunkt nicht differenzierbar ist, müssen
Wege wieder den Nullpunkt meiden.
Da die analytische Fortsetzung stetig ist, muss der Logarithmuswert
\[
\log\gamma(t)
=
\log|\gamma(t)| + i\varphi(t)
\]
sein.
In jedem Punkt gibt es, konsistent mit
\eqref{buch:analytisch:fortsetzung:eqn:logfort},
unendlich viele mögliche Werte.
Der Wert der analytischen Fortsetzung hängt von der Anzahl der
Umrundungen des Nullpunktes durch den Weg ab.

Wie im Falle der Wurzelfunktion lässt sich die Logarithmusfunktion
also auf $D=\mathbb{C}\setminus\{0\}$ nicht widerspruchsfrei
definieren.
Dazu muss der Definitionsbereich durch die Menge
\[
\mathbb{R}^+\times \mathbb{R}
=
\{
(r,\varphi)
\mid
r>0
\}
\]
ersetzt werden, der durch die Abbildung
\[
\pi
\colon
(r,\varphi)
\mapsto
r(\cos\varphi+i\sin\varphi)
\]
auf $D$ abgebildet werden.
Über jedem Punkt $z\in D$ sitzen unendlich viele Punkte $p$ mit
$\pi(p)=z$, die zu den verschiedenen möglichen Werten des Logarithmus
gehört.

%
% Universelle Überlagerung von $\mathbb{C}\setminus\{0\}$
%
\subsubsection{Universelle Überlagerung von $\mathbb{C}\setminus\{0\}$}
Die beiden Beispiele der Quadratwurzel und der Logarithmusfunktion
zeigen zwar, dass es jeweils möglich ist, einen Definitionsbereich
zu konstruieren, auf dem sich die Funktion widerspruchsfrei definieren
lässt.
Für die Konstruktion wird die analytische Fortsetzung ausgehend von 
einem festen Punkt verwendet.
In beiden Beispielen kam es darauf an, wie oft der Weg, entlang dem
die analytische Fortsetzung gebildet wird, den Nullpunkt umrundet.
Auf die Details des Weges kommt es nicht an.
Zwei Wege mit den gleichen Endpunkten, die sich ineinander deformieren
lassen, also homotop sind,
\index{homotop}%
führen auf den gleichen Funktionswert.

Für ein allgemeine Definition eines Kandidaten des Definitionsbereichs
müssen also homotope Pfade zwischen Punkten $p_0$ und $p_1$ innerhalb
eines Gebiets $B\subset\mathbb{C}$ untersucht werden.
Wir parametrisieren Pfade mit Parametern aus dem Einheitsintervall
$I=[0,1]$.
Eine Homotpie $H$ zwischen zwei Pfaden in $B$ ist ein Abbildung
\[
H\colon I\times I\to B
\qquad\text{mit}\qquad
\left\{
\begin{aligned}
H(t,0) &= \gamma(t) \\
H(t,1) &= \delta(t) \\
H(0,s) &= p_0 \\
H(1,s) &= p_1 
\end{aligned}
\right.
\]
Die ersten zwei Bedingungen besagen, dass die Homotopie die Kurve
$\gamma$ in die Kurve $\delta$ deformiert, wenn der Parameter $s$
von $0$ in $1$ variert wird.
Die letzten zwei Bedingungen besagen, dass die Endpunkte der Kurven
während der Deformation mit dem Parameter $s$ fest bleiben.
Die Menge der zu $\gamma$ homotopen Pfade wird mit $[\gamma]$
bezeichnet.

\begin{definition}[Universelle Überlagerung]
Die Menge
\[
E
=
\{
[\gamma]
\mid
\gamma\colon I\to B
\text{ mit }
\gamma(0)=p_0
\}
\]
mit der Abbildung
\[
\pi
\colon
E\to B
:
[\gamma] \mapsto \gamma(1)
\]
heisst die
\emph{universelle Überlagerung}
\index{universelle Uberlagerung@universelle Überlagerung}%
\index{Uberlagerung, universelle@Überlagerung, universelle}%
von $B$.
Die Abbildung $\pi$ heisst die \emph{Projektion}.
\index{Projektion}
\end{definition}

\begin{satz}
Die Menge $E$ ist zusammenziehbar.
\end{satz}

\begin{proof}
TODO
\end{proof}

%
% Die Fundamentalgruppe
%
\subsubsection{Die Fundamentalgruppe}
Die Pfade in der Menge $\pi_1(B)=\pi^{-1}(p_0)$ können durch die
Verkettung
\[
\gamma\cdot \delta
\colon
I\to B
:
t
\mapsto
(\gamma\cdot\delta)(t)
=
\begin{cases}
\gamma(2t)&\qquad \text{für $0\le t\le \frac12$}\\
\delta(2t-1)&\qquad \text{für $\frac12\le t\le 1$}
\end{cases}
\]
miteinander verknüpft werden.
Homotope Kurven haben homotope Verknüpfung.

Die Verknüpfung der Homotopieklassen von Pfaden ist assoziativ.
Seien $\gamma_1$, $\gamma_2$ und $\gamma_3$ drei Pfade mit Endpunkten
$p_0$.
Dann ist
\begin{align*}
(\gamma_1\cdot\gamma_2)\cdot\gamma_3
&\colon
I
\to
B
:
t
\mapsto
\begin{cases}
\gamma_1(4t)&\qquad\text{für $0\le t\le \frac14$}\\
\gamma_2(4t-1)&\qquad\text{für $\frac14\le t\le \frac12$}\\
\gamma_3(2t-1)&\qquad\text{für $\frac12\le t\le 1$}
\end{cases}
\\
\gamma_1\cdot(\gamma_2\cdot\gamma_3)
&\colon
I
\to
B
:
t
\mapsto
\begin{cases}
\gamma_1(2t)&\qquad\text{für $0\le t\le \frac12$}\\
\gamma_2(4t-2)&\qquad\text{für $\frac12\le t\le \frac34$}\\
\gamma_3(4t-3)&\qquad\text{für $\frac34\le t\le 1$}
\end{cases}
\end{align*}
Die Abbildung
\[
H\colon I\times I \to B
:
(t,s)
\mapsto
\begin{cases}
\gamma_1((4-2s)t)     &\text{für $0\le t\le \frac14(1+s)$}\\
\gamma_2(4t-1-s)      &\text{für $\frac14(1+s) \le t \le \frac14(2+s) $}\\
\gamma_3((2+2s)t-1-2s &\text{für $\frac14(2+s) \le t \le 1$}\\
\end{cases}
\]
ist eine Homotopie zwischen 
$(\gamma_1\cdot\gamma_2)\cdot\gamma_3$
und
$\gamma_1\cdot(\gamma_2\cdot\gamma_3)$.
Damit ist gezeigt, dass die Verkettung von Wegen bis auf Homotopie
assoziativ ist.

Der konstante Pfad $e(t) = p_0$ erfüllt
$\gamma\cdot e=e\cdot \gamma = \gamma$, z.~B.~ist 
\[
H
\colon
I\times I \to B
:
(t,s)
\mapsto
\begin{cases}
\gamma((1+s)t) &\qquad\text{für $0\le t \le \frac12(1+s)$}\\
p_0            &\qquad\text{für $\frac12(1+s)\le t\le 1$}
\end{cases}
\]
ein Homotpie zwischen $\gamma$ und $\gamma\cdot e$.
Ein ähnliche Homotopie kann konstruiert werden zwischen $e\cdot\gamma$
und $\gamma$.
Somit ist der konstante Pfad bis auf Homotpie ein neutrales Element 
für die Verkettung von Pfaden.

Zu $\gamma\in \pi_1(B)$ ist
\[
\gamma^{-1}
\colon
I\to B
:
t\mapsto \gamma^{-1}(t) = \gamma(1-t)
\]
der inverse Pfad mit der Eigenschaft.
Tatsächlich ist
\[
H
\colon
I\times I
\to
B
:
(t,s)
\mapsto
\begin{cases}
\gamma((1-s)2t)    &\qquad\text{für $0\le t\le \frac12$}\\
\gamma((1-s)(1-2t))&\qquad\text{für $\frac12\le t\le 1$}
\end{cases}
\]
eine Homotpie zwischen $\gamma\cdot\gamma^{-1}$ und dem konstanten
Weg $e$.
Damit ist gezeigt, dass die Menge $\pi_1(B)$ eine Gruppe ist.

\begin{definition}[Fundamentalgruppe]
Die Menge $\pi_1(B)=\pi^{-1}(p_0)$ heisst die Fundamentalgruppe von $B$.
\end{definition}

\begin{satz}
Ist $B$ ein zusammenziehbarer Raum, dann ist die Fundamentalgruppe
$\pi_1(B)=\{e\}$ trivial.
\end{satz}

\begin{proof}
Wenn $H\colon B\times I\to B$ eine Homotopie zwischen der identischen
Abbildung $\operatorname{id}_B\colon B\to B$ und der konstanten Abbildung
$e(b) = p_0$ und $\gamma$ eine geschlossene Kurve in $B$, dann ist
\[
I\times I
\to
B
:
(t,s)
\mapsto
H(\gamma(t),s)
\]
eine Homotopie zwischen der Kurve $\gamma$ und der konstanten Kurve $e$.
Jede Kurve ist damit homotop zum neutralen Element in der Fundamentalgruppe.
\end{proof}


