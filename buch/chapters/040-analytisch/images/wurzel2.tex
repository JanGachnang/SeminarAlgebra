%
% wurzel2.tex -- template for standalon tikz images
%
% (c) 2021 Prof Dr Andreas Müller, OST Ostschweizer Fachhochschule
%
\documentclass[tikz]{standalone}
\usepackage{amsmath}
\usepackage{times}
\usepackage{txfonts}
\usepackage{pgfplots}
\usepackage{csvsimple}
\usetikzlibrary{arrows,intersections,math}
\begin{document}
\def\skala{6}
\def\kurvenradius{0.6}
\pgfmathparse{1.8*180}
\xdef\kurvemax{\pgfmathresult}
\def\kurvesteps{18}
\def\patchradius{0.25}
\definecolor{kurvenfarbe}{rgb}{0.9,0.9,0}
\definecolor{patchfarbe}{rgb}{0.8,0.2,0.4}
\definecolor{schnittfarbe}{rgb}{0.2,0.2,1.0}
\begin{tikzpicture}[>=latex,thick,scale=\skala]

% add image content here

\draw[->] (-0.95,0) -- (0.95,0) coordinate[label={$\operatorname{Re}$}];
\draw[->] (0,-0.95) -- (0,0.95) coordinate[label={right:$\operatorname{Im}$}];
\foreach \k in {0,...,\kurvesteps}{
	\pgfmathparse{\k*(\kurvemax/\kurvesteps)}
	\xdef\winkel{\pgfmathresult}
	\fill[color=patchfarbe!20,opacity=0.5]
		(\winkel:\kurvenradius) circle[radius=\patchradius];
	\draw[color=patchfarbe]
		(\winkel:\kurvenradius) circle[radius=\patchradius];
}

\draw[color=kurvenfarbe,line width=2pt]
	(0:\kurvenradius) arc(0:\kurvemax:\kurvenradius);

\foreach \k in {0,...,\kurvesteps}{
	\pgfmathparse{\k*(\kurvemax/\kurvesteps)}
	\xdef\winkel{\pgfmathresult}
	\fill[color=patchfarbe]
		(\winkel:\kurvenradius) circle[radius=0.01];
}

\begin{scope}
	\clip (0:\kurvenradius)
		circle[radius=\patchradius];
	\fill[color=schnittfarbe!20] (\winkel:\kurvenradius)
		circle[radius=\patchradius];
\end{scope}

\node at (-0.8,0.8) {$\mathbb{C}$};

\end{tikzpicture}
\end{document}

